%%
\section{Formul\'acia \'ulohy}

Cieľom tejto diplomovej práce je analyzovať, navrhnúť, implementovať a otestovať aplikačný rámec 
pre problém sprostredkovania správ v~protokole IPFIX. Riešenie je zároveň potrebné integrovať s~
existujúcou architektúrou nástroja SLAmeter. Za týmto účelom je nutné vykonať nasledujúce kroky. 

V~prvom rade je potrebné analyzovať IPFIX z~pohľadu protokolu ale aj architektúry.
Najväčší dôraz sa kladie na analýzu správ, pretože 
úlohou Mediátora bude aj ich dekódovanie a následné zakódovanie podľa špecifikovaného formátu.

Druhým krokom je analýza problému sprostredkovania správ v~IPFIX. Konkrétne ide o~definovanie terminológie,
analýzu výhod a príkladov použitia, ale aj priblíženie niektorých problémov spojených s~implementáciou 
takéhoto nástroja.

Následne je potrebné čitateľovi aspoň stručne priblížiť projekty skupiny MONICA, medzi ktoré patrí
BasicMeter a jeho nadstavba SLAmeter. 

Štvrtým a najdôležitejším krokom je navrhnúť a implementovať samotný aplikačný rámec podľa definovaných 
požiadaviek. Riešenie musí byť experimentálne overené vhodnými testami.

Posledným, no nie menej dôležitým krokom je podľa pokynov vypracovať dokumentáciu vykonanej práce.

%Text záverečnej práce musí obsahovať\/ kapitolu s~formuláciou
%úlohy resp. úloh riešených v~rámci záverečnej práce. V~tejto časti
%autor rozvedie spôsob, akým budú riešené úlohy a~tézy formulované
%v~zadaní práce. Taktiež uvedie prehľad podmienok riešenia.