\setcounter{page}{1}
\setcounter{equation}{0}
\setcounter{figure}{0}
\setcounter{table}{0}

\section*{\'Uvod}
\addcontentsline{toc}{section}{\numberline{}\'Uvod}

Rozmach počítačových sietí prináša ľudstvu nové možnosti a služby, o akých sme kedysi ani nesnívali.
Internet sa stal súčasťou nášho každodenného života, mnohí sú pripojení aj 24 hodín denne, nielen 
prostredníctvom počítačov, ale aj mobilných zariadení. Práve vďaka rozmachu mobilného Internetu si už
ani nevieme predstaviť situáciu, že by sme neboli v dosahu Internetového obsahu. Týmto sú kladené 
obzvlášť vysoké nároky na kvalitu sietí a poskytovaných služieb. Aby poskytovatelia Internetu mohli 
zabezpečovať a zlepšovať kvalitu služieb (QoS), musia ju vedieť merať.

V Laboratóriu počítačových sietí \emph{(CNL)} na Technickej Univerzite v Košiciach vyvíja výskumná 
skupina MONICA pasívny merač parametrov sieťovej prevádzky na báze tokov, ktorý vyhodnocuje 
dodržiavanie zmluvy o úrovni poskytovanej služby \emph{(SLA)}. Jeho cieľom je spracovať určité parametre 
sieťovej prevádzky a na ich základe vypočítať triedu kvality. Táto aplikácia má slúžiť nielen poskytovateľom 
pripojenia k Internetu ako monitorovací nástroj. Triedy umožňujú aj ich klientom jednoducho 
porovnávať kvalitu poskytovaných služieb. Jadro nástroja SLAmeter sa skladá z komponentov, ktoré sú 
navrhnuté v súlade s protokolom IPFIX.

Komponentmi architektúry IPFIX \emph{(IP Flow Information Export)} podla RFC 5470 \citep{rfc5470}
sú exportéry a kolektory komunikujúce protokolom IPFIX. Vzhľadom k trvalému rastu IP prevádzky
v heterogénnych sieťových prostrediach, tieto exportér-kolektor systémy môžu viesť k problémom 
škálovateľnosti. Ba čo viac, neposkytujú flexibilitu potrebnú pre široký rad meracích aplikácii.

Pre naplnenie požiadaviek aplikácií s obmedzenými systémovými zdrojmi, IPFIX architektúra zavádza 
sprostredkovateľskú entitu medzi exportéry a kolektory. Z pohľadu manipulácie s dátami môžu sprostredkovateľské
moduly tejto entity poskytovať agregáciu, koreláciu, filtrovanie, selekciu, anonymizáciu a iné úpravy záznamov o 
tokoch za účelom 
šetrenia výpočtových zdrojov meracieho systému a vykonávania predspracovania úloh pre kolektor. Z hľadiska
interoperability nástrojov rôznych vývojárov, môžu poskytovať konverziu z iných protokolov na IPFIX, 
respektíve zvyšovať spoľahlivosť exportov napríklad prevodom z nespoľahlivého, bezspojovo orientovaného 
protokolu UDP na spoľahlivý SCTP. \citep{rfc6183}

Táto práca analyzuje nastolený problém sprostredkovania správ a sprostredkovateľských entít nazývaný 
\emph{IP Flow Information Export (IPFIX) Mediation Problem}. Jej hlavným cieľom je navrhnúť a implementovať
aplikačný rámec pre sprostredkovateľskú entitu zvanú IPFIX Mediátor pre účely nástroja SLAmeter. 
Rámec musí klásť veľký dôraz na jeho modularitu, aby bolo jednoduché a pohodlné implementovať nové 
sprostredkovateľské moduly a tým zvyšovať možnosti monitorovania aplikáciou SLAmeter.

Štruktúra práce je nasledovná. V úvodnej kapitole je formulovaná definícia úlohy. Druhá kapitola sa 
v krátkosti venuje protokolu IPFIX. Popisuje len formát IPFIX správy, jej hlavičku, tri typy
sád a ich komponenty. Ostatné rysy protokolu si môže čitateľ vyhľadať v príslušných dokumentoch RFC.
Nasledujúca kapitola podrobne analyzuje problém sprostredkovania správ v IPFIX. Úvodom definuje použitú 
terminológiu. Analyzuje nevýhody exportér-kolektor architektúr bez Mediátora, príklady použitia 
sprostredkovania, ale aj jeho implementačno-špecifické problémy. Ďalšia sekcia sa zaoberá analýzou 
aplikačného rámca pre IPFIX Mediátor. Tretia kapitola stručne priblíži projekty výskumnej skupiny MONICA.
Nasledujúca kapitola s poradovým číslom štyri, tvorí jadro práce. V nej sa čitateľ dočíta o návrhu a 
implementácii aplikačného rámca pre sprostredkovateľskú entitu. Jednotlivým triedam a metódam vyššieho 
významu je venované podrobné vysvetlenie. Predposledná kapitola niekoľkými testami experimentálne 
overuje implementované riešenie. Záver už len zhodnotí diplomovú prácu a jej dosiahnuté výsledky. 


%V~úvode autor podrobnejšie ako v~predhovore, pritom výstižne a~krátko
%charakterizuje stav poznania alebo praxe v~špecifickej oblasti, ktorá
%je predmetom záverečnej práce. Autor presnejšie ako v~predhovore
%vysvetlí ciele práce, jej zameranie, použité metódy a~stručne objasní
%vzťah práce k~iným prácam podobného zamerania. V~úvode netreba
%zachádzať hlbšie do teórie. Nie je potrebné podrobne popisovať metódy,
%experimentálne výsledky, ani opakovať závery prípadne odporúčania,
%pozri~\citep{kat}.