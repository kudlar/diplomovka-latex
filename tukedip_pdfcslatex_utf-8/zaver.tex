\section{Zhodnotenie rie\v{s}enia}

Najdôležitejším cieľom tejto diplomovej práce bolo navrhnúť a implementovať aplikačný rámec pre 
sprostredkovateľskú entitu (IPFIX Mediátor), ktorá by bola medzičlánkom v~komunikácii medzi exportérom
BEEM a kolektorom JXColl nástroja SLAmeter. Mediátor má poskytovať rozhranie pre manipuláciu so 
sprostredkovateľskými modulmi, ktoré budú poskytovať rôzne funkcie na modifikáciu IPFIX správ ešte pred ich 
spracovaním v~kolektore. Práca túto úlohu aj splnila, čo bolo potvrdené v~poslednej časti 
práce venovanej experimentálnemu overeniu.

Úvodom sa čitateľ oboznámil s~analýzou formátu IPFIX správ rovnako ako s~výhodami použitia 
IPFIX Mediátora v~meracej architektúre. Samostatné časti boli venované príkladom použitia, ale aj 
úskaliam spojeným s~implementáciou tohto nástroja. Nechýbalo ani stručné predstavenie projektov výskumnej 
skupiny MONICA.

Najvýznamnejším výsledkom práce je funkčná implementácia aplikačného rámca v~jazyku Java. Jeho návrh je 
odpoveďou na definované požiadavky a analyzované implementačno-špecifické problémy. Zhromažďovací proces 
bol prispôsobený, no vychádzal z~existujúcej implementácie kolektoru JXColl. Dôležitou častou práce je 
podpora pre sprostredkovateľské procesy. Tok dát medzi modulmi, či už sériový, alebo paralelný sa  
jednoducho nastavuje v~konfiguračnom XML súbore podľa navrhnutého formátu. Na základe tejto konfigurácie 
riadi dispečer distribúciu záznamov o~tokoch medzi procesmi. Definované moduly sa však najprv musia 
pospúšťať. Toto zabezpečila pokročilá technológia dynamického načítavania tried pomocou 
systémového \emph{class loadera} a reflexie. Ďalším pokročilým riešením bolo zabezpečenie toho, že 
všetky sprostredkovateľské moduly musia mať len jednu unikátnu inštanciu. Keďže dizajn jazyka Java 
nepovoľuje \emph{abstract static} metódy, bolo potrebné využiť návrhový vzor \emph{Factory method} a 
hash mapu uchovávajúcu existujúce inštancie modulov. 

Úlohou aplikačného rámca je aj poskytnúť metódy na dekódovanie a zakódovanie hodnôt informačných elementov.
Pri zakódovaní bol kladený veľký doraz na efektivitu a výpočtový výkon. Preto navrhnuté metódy namiesto 
využívania tried jazyka pracujú na úrovni bitových operácii. Budúci riešitelia určite ocenia prítomnosť
vzorového sprostredkovateľského modulu, ktorý demonštruje navrhnuté zásady programovania modulov a 
využíva všetky dostupné metódy, ktoré mu poskytuje abstraktná rodičovská trieda. 

Implementácia bola nakoniec overená niekoľkými experimentmi, ktoré dokázali funkčnosť riešenia. Ďalšie 
smerovanie nástroja predpokladá implementovanie sprostredkovateľských modulov, ktoré rozšíria monitorovacie 
možnosti aplikácie SLAmeter. Najväčší prínos sa vidí v~anonymizačnom module, konverznom module z~protokolu 
NetFlow na IPFIX, prípadne v~module, ktorý
bude na základe istých pravidiel robiť selekciu IPFIX správ a tie následne exportovať distribuovaným 
kolektorom. Z~pohľadu aplikačného rámca by bolo vhodné rozšíriť jeho zhromažďovací proces o~podporu 
ďalších transportných protokolov ako TCP a SCTP. Rovnaké rozšírenie je možné aj na strane exportovacieho 
procesu.


%Táto časť\/ záverečnej práce je povinná. Autor uvedie zhodnotenie
%riešenia. Uvedie výhody, nevýhody riešenia,  použitie výsledkov, ďalšie
%možnosti a~pod., prípadne načrtne iný spôsob riešenia úloh, resp.
%uvedie, prečo postupoval uvedeným spôsobom.